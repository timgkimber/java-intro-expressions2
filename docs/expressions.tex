\documentclass[11pt,a4paper]{article}
\usepackage{java-specs}

\DeclareMathOperator*{\binop}{\#}

\module{536: Introduction to Java}
\classes{MSc in Computing Science}
\title{Tutorial 1 --- Expressions}
\date{}


\begin{document}
 
\maketitle


In this exercise you will continue to implement Java types representing
simple arithmetic expressions.
You can either use the code you have written yourself so far,
or start from the code provided.

\section{Aims}

This tutorial will help you understand how to: 

\begin{itemize}
  \item Write simple Java classes
  \item Implement inheritance in Java
  \item Use abstract classes in Java
  \item Make use of types from the Java API
\end{itemize}

 
\section{Your Development Environment}
 
Something you might want to think about at this point is the sort of
development environment you want to use.
In the lecture I have been using command line tools and if you like things
like vim and make, then feel free to stick to that.
For Java programming, however,
an \emph{integrated development environment} (IDE) is a very popular 
alternative.
An IDE is a sophisticated graphical tool that is designed to help:

\begin{itemize}
  \item Organise your code, including via version control
  \item Compile and run your code
  \item Debug and correct your code
  \item Refactor your code
  \item Build and test your application
\end{itemize}

Two of the most popular IDEs, IntelliJ and Eclipse, are both available on the 
lab machines (under Programming on the Applications menu).
For what it's worth, I always use the Eclipse IDE, certainly for bigger projects.
Although, I am told that the most recent versions of IntelliJ are thought to be
better.
My feeling is that doing everything on the command line requires a lot of
manual set up that I prefer to avoid.
This is very much a personal view though --- others will disagree.
If you want to try using an IDE then have a go.
Look for a tutorial online.
There will be an initial learning period,
but once you get the hang of it you will speed up no end.

\section{Obtain Provided Files}

Download the provided code for this tutorial from CATE into a suitable
directory and unzip the file.

\section {What To Do}

\subsection{Implement Remaining Expressions}

Your first task is to implement two other types of expression: a sum (addition)
and a product (multiplication).
These should be implemented as the classes \java{Sum} and \java{Product}
as follows:

\begin{itemize}
  \item Both classes should be in the same package as \java{WholeNumber}
  \item Both classes should have a constructor that accepts two 
    \java{Expressions}.
    In each case, the first argument is the left operand and the second
    is the right operand.
  \item Both classes should implement the \java{Expression} interface.
  \item Both classes should override the \java{toString()} method of the 
  superclass \java{Object}.
\end{itemize}


\subsection{Refactor Expression Types}

The two classes you have just written share various properties:

\begin{itemize}
  \item They have the same member variables
  \item Their constructors are identical
\end{itemize}

So, you can extract this part of the classes, 
the part that defines their state, into a common parent class 
\java{BinaryExpression}.
\java{BinaryExpression} should declare that it implements \java{Expression},
so that all \java{BinaryExpression}s are automatically \java{Expressions}.
This might seem a bad choice, because you cannot define 
a single way to evaluate all binary expressions.
To find out how to solve this problem, and how to implement subclasses in Java
you will need to go to the Java Tutorial and doing a bit of reading.

The main section you need is on inheritance, under 
``Interfaces and Inheritance'':

https://docs.oracle.com/javase/tutorial/java/IandI/subclasses.html

There are several subsections under that page (see the navigation on the left).
The important ones to look at now are ``Using the Keyword super'',
so you can see how to call the constructor of a superclass from a subclass,
and ``Abstract Methods And Classes''.
Make sure you read the very last section about abstract classes, 
when an abstract class implements an interface.
Finally, you will need to think about the visibility of the fields
of \java{BinaryExpression}.
This is covered in the section on ``Controlling Access to Members of a Class'':

https://docs.oracle.com/javase/tutorial/java/javaOO/accesscontrol.html

You should limit the visibility of the fields as much as possible
while still allowing the access you need.


\subsection{Sorting Expressions}

You have seen how to define and implement your own interface type in Java.
The Java API also contains ``off the shelf'' interfaces that your classes
can implement. 
Doing this bestows on the class some useful ability.
One such interface is \java{java.lang.Comparable}.
You can read its documentation at:

https://docs.oracle.com/javase/8/docs/api/java/lang/Comparable

In a nutshell, objects of any class implementing \java{Comparable}  
can be sorted using \java{Arrays.sort(...)} and other similar methods.

Your next task is to implement \java{Comparable} for all \java{Expression}
types, so that they are sortable.
Firstly, make \java{Expression} a subtype of \java{Comparable}.
Interfaces can inherit from one another like classes,
using the same notation.
So, change the declaration of \java{Expression} to be:

\begin{itemize}
  \item \java{public interface Expression 
    extends Comparable<Expression> \{ ... \}}
\end{itemize}

The \java{<Expression>} is necessary because \java{Comparable} is
\emph{generic}.
Don't worry about this now.
You will learn more about Java generics later in the course.
For now you need to decide how you are going to implement
\java{Comparable<Expression>}.
The interface has a single method: 

\begin{itemize}
  \item \java{public int compareTo(Expression other)}
\end{itemize}

which must be defined for all \java{Expression}s.
Decide in which class or classes this method should be defined.
A call \java{this.compareTo(that)} returns:

\begin{itemize}
  \item a positive int if \java{this} is ``greater than'' \java{that}
  \item a negative int if \java{this} is ``less than'' \java{that}
  \item zero if \java{this} is ``equal to'' \java{that}
\end{itemize}

You should use the value of the expression (the integer value obtained by
evaluating the expression) to determine the order.
Test your implementation by running \java{ExpressionProg}.

\section{And Finally}

If you have completed the tutorial before the end of the session then
use the time to make some notes on Java.
You will almost certainly think of questions that have not been answered 
by the course.
In this case, try to work out the answers for yourself by experimenting
with the programs from the lectures or devising your own examples.
If you get stuck the Java Tutorial is a very good resource. 








% \subsection*{OO Implementation}
% 
% Expressions can be implemented using the following class hierarchy.
% 
% \begin{figure}[tbh]
%   \includegraphics[scale=0.6]{images/uml}
%   \caption{The \java{Expression} classes.}
% \end{figure}
% 
% 
% \begin{itemize}
%   \item An \java{Expression} is an object with a value and a precedence.
%         The class \java{Expression} will not possess enough information to be
%         instantiated directly. 
%         It will be an abstract super class. 
%   \item A \java{NumberExpression} is a kind of \java{Expression} that 
%         encapsulates an integer value.
%   \item An \java{LBinaryExpression} is a kind of \java{Expression} 
%         that represents a left-associative binary operation between 
%         two sub-expressions. 
%         It has a left operand, an operator symbol, and a right operand.
%         An operation $\binop$ is left-associative if $E_1 \binop E_2 \binop E_3$
%         means $(E_1 \binop E_2) \binop E_3$.
%   \item \java{Addition} and \java{Multiplication} are two kinds of 
%         \java{LBinaryExpression}.
%         The symbol of an \java{Addition} is `` + '', 
%         and the symbol of a \java{Multiplication} is `` * ''.
%   \item All \java{Expression}s can be evaluated. 
%         The evaluation result is the same as that of the arithmetic expression
%         it represents.
%   \item All \java{Expression}s can be turned to a string. 
%         This string is the text representation of the expression, 
%         using minimum brackets.
%         Brackets are required only when the precedence of some operand 
%         expression is \java{lower} than the precedence of the binary 
%         expression of which it is a part.
%         See the implementation of this feature in the C++ code below.
%         
% \end{itemize}
% 
%  
% \section*{What To Do}
% 
% You have been given the code for \java{Expression.java}.
% Write Java implementations of the other four classes described above.
% A C++ implementation is shown below, for reference.
% 
% \begin{itemize}
%   \item Save \java{Expression.java} in a suitable directory and add more
%         source files in the same place.
%         Remember each Java class \java{NameOfClass} should have its own
%         file.
%         This file must be named \java{NameOfClass.java} or your code will not 
%         compile.
%   \item Start by implementing \java{NumberExpression}.
%         This is a non-abstract class that should subclass \java{Expression}.
%   \item Note that, in the Java implementation,
%         all printed output is generated by the 
%         (non-abstract, static) \java{Expression.printExpression} method.
%         This (implicitly) calls the expression's \java{toString()} method
%         to obtain its string representation.
%         The \java{toString()} method is inherited from \java{Object} and should
%         be overridden in the different expression classes.
%   \item \java{Expression} has been given a \java{main} method that creates
%         three \java{NumberExpression}s and prints them out.
%         Look at \java{main} to see how \java{NumberExpression}s are constructed.
%         Once your \java{NumberExpression} is complete, you should be able to
%         compile both files with
% %
%         \begin{verbatim}
%   javac Expression.java
%         \end{verbatim}
%         and then run \java{Expression.main} with
%         \begin{verbatim}
%   java Expression
%         \end{verbatim}
% %        
%   \item Now implement the binary expression classes.
%         Uncomment the rest of \java{Expression.main} to test as you go.
%         
%         
% \end{itemize}
% 
% \section*{C++ Code}
% 
% %\begin{center}
% \raggedright
% \includegraphics[scale=0.9]{images/expcpp}
% \includegraphics[scale=0.9]{images/numcpp}
% \includegraphics[scale=0.9]{images/bincpp}
% \includegraphics[scale=0.9]{images/addcpp}
% \includegraphics[scale=0.9]{images/mulcpp}
%    
%\end{center}

\end{document}



